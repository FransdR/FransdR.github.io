% File for idea dualization of two stage problems
\documentclass[11pt]{article}
\usepackage{amsmath,amsthm,amssymb}
\usepackage{booktabs}
\usepackage{hyperref}
\usepackage{algorithm,algorithmic}
  \renewcommand{\algorithmicrequire}{\textbf{Input:}}
  \renewcommand{\algorithmicensure}{\textbf{Output:}}
\usepackage[backend=bibtex,
      natbib=true,
			useprefix=true,
			url=false,
			isbn=false,
			doi=false,
			maxbibnames=99,
			maxnames=99,
			citetracker=true,
			firstinits=true,
			style=authoryear]{biblatex}    %needed to use authors name in references

	\AtEveryCitekey{\ifciteseen{}{\defcounter{maxnames}{3}}}

	\usepackage[american]{babel}				%Recommended for biblatex
	\usepackage{csquotes}				%Recommended for biblatex
	%The bibliography:
	    \addbibresource{SuperMasterbib.bib}
      \addbibresource{OptimalLearningBibFile.bib}
	\renewbibmacro{in:}{} %Surpress In: in the reference sheet
	\renewbibmacro*{volume+number+eid}{%
  \printfield{volume}%
%  \setunit*{\adddot}% DELETED
  \setunit*{\addnbspace}% NEW (optional); there's also \addnbthinspace
  \printfield{number}%
  \setunit{\addcomma\space}%
  \printfield{eid}}
\DeclareFieldFormat[article]{number}{\mkbibparens{#1}}
\DeclareFieldFormat
  [article,inbook,incollection,inproceedings,patent,thesis,unpublished]
  {title}{{#1\isdot}}

\newtheorem{theorem}{Theorem}
\newtheorem{property}{Property}
\newtheorem{example}{Example}

% New commands
\newcommand{\set}[1]{\mathcal{#1}}
\newcommand{\R}{\mathbb{R}}
\newcommand{\vect}[1]{\mathbf{#1}}
\newcommand{\matr}[1]{\mathbf{#1}}
\newcommand{\relint}[1]{\text{ri}(#1)}

\usepackage{titling}
\setlength{\droptitle}{-10em}   % This is your set screw

\begin{document}
\title{Model formulation for the facility location model}
\date{}
\author{Frans de Ruiter}
\maketitle
\thispagestyle{empty}

\subsection*{Parameters}
Let $i=1,\dots,N$ be the index for the cities.\\ \\
\begin{tabular}{rl}
  $d_{i}$ : & demand at city $i$ in number of pallets\\
  $F_i$ : & fixed costs for opening a distribution center (DC) at location $i$ in euros\\
  $S_i$ : & capacity at DC $i$ (if opened) in number of pallets\\
  $c_{ij}$ : & costs for transporting one pallet from city $i$ to city $j$ in euros\\
\end{tabular}

\subsection*{Variables}
\begin{tabular}{rl}
  $x_i$ : & $=\begin{cases} 1 \text{ if a DC is opened in city } i\\ 0 \text{ otherwise} \end{cases}$\\
  $y_{ij}$ : & number of pallets transported from city $i$ to city $j$
\end{tabular}

\subsection*{Model}
\begin{align*}
\begin{aligned}
  & \min_{x,y} & & \sum_{i=1}^N F_ix_i + \sum_{i=1}^N\sum_{j=1}^Nc_{ij}y_{ij}\\
  & \text{s.t.}
        & & \sum_{i=1}^N y_{ij} = d_{j} \qquad j=1,\dots,N\\
  & & & \sum_{j=1}^N y_{ij} \leq S_ix_i \qquad i=1,\dots,N\\
  & & & y_{ij} \geq 0 \qquad i=1,\dots,N,\ j=1,\dots,N\\
  & & & x_i \in \{0,1\} \qquad i=1,\dots,N.
\end{aligned}
\end{align*}\\
The first constraint ``$\sum_{i=1}^N y_{ij} = d_{j}$'' ensures that the demand in the $i$-th city is met. The constraint ``$\sum_{j=1}^N y_{ij} \leq S_ix_i$'' ensures that the total number of pallets leaving city $i$ is less than $0$ if there is no DC opened in city $i$ and less than $S_i$ if a DC is opened. Nonnegativity of the transport amount is guaranteed by the constraints ``$y_{ij} \geq 0$'' and the binary nature (open or closed) of a DC by ``$x_i \in \{0,1\}$''.
% \begin{abstract}
%     A fast way of adaptability via sampling via the dual formulation. This is explained for the linear case, it works the same for nonlinear problems.
% \end{abstract}


\end{document}
